% Document template for ANS Journals
% Options: footnoteAtEnd - Places all footnotes at the end of document
%               Usage: \documentclass[footnoteAtEnd]{style/nseJournal}
\documentclass{style/nseJournal}
\usepackage{diagbox}
\usepackage{bm}

% Ways of grouping things
\newcommand{\bracket}[1]{\left[ #1 \right]}
\newcommand{\bracet}[1]{\left\{ #1 \right\}}
\newcommand{\fn}[1]{\left( #1 \right)}
\newcommand{\ave}[1]{\left\langle #1 \right\rangle}

% Derivative forms
\newcommand{\dx}[1]{\,d#1}
\newcommand{\dxdy}[2]{\frac{\partial #1}{\partial #2}}
\newcommand{\dxdt}[1]{\frac{\partial #1}{\partial t}}
\newcommand{\dxdz}[1]{\frac{\partial #1}{\partial z}}
\newcommand{\dfdt}[1]{\frac{\partial}{\partial t} \fn{#1}}
\newcommand{\dfdz}[1]{\frac{\partial}{\partial z} \fn{#1}}
\newcommand{\ddt}[1]{\frac{\partial}{\partial t} #1}
\newcommand{\ddz}[1]{\frac{\partial}{\partial z} #1}
\newcommand{\dd}[2]{\frac{\partial}{\partial #1} #2}
\newcommand{\ddx}[1]{\frac{\partial}{\partial x} #1}
\newcommand{\ddy}[1]{\frac{\partial}{\partial y} #1}

% Vector forms
\renewcommand{\vec}[1]{\overrightarrow{#1}}
\renewcommand{\div}{\vec{\nabla}\! \cdot \!}
\newcommand{\grad}{\vec{\nabla}}
\newcommand{\oa}[1]{\fn{\frac{1}{3}\hat{\Omega}\!\cdot\!\overrightarrow{A_{#1}}}}

% Equation beginnings and endings
\newcommand{\bea}{\begin{eqnarray}}
\newcommand{\eea}{\end{eqnarray}}
\newcommand{\be}{\begin{equation}}
\newcommand{\ee}{\end{equation}}
\newcommand{\beas}{\begin{eqnarray*}}
	\newcommand{\eeas}{\end{eqnarray*}}
\newcommand{\bdm}{\begin{displaymath}}
\newcommand{\edm}{\end{displaymath}}

% Equation punctuation
\newcommand{\pec}{\hspace{0.25in},}
\newcommand{\pep}{\hspace{0.25in}.}
\newcommand{\pev}{\hspace{0.25in}}

% Equation labels and references, figure references, table references
\newcommand{\LEQ}[1]{\label{eq:#1}}
\newcommand{\EQ}[1]{Eq.~(\ref{eq:#1})}
\newcommand{\EQS}[1]{Eqs.~(\ref{eq:#1})}
\newcommand{\REQ}[1]{\ref{eq:#1}}
\newcommand{\LFI}[1]{\label{fi:#1}}
\newcommand{\FI}[1]{Fig.~\ref{fi:#1}}
\newcommand{\RFI}[1]{\ref{fi:#1}}
\newcommand{\LTA}[1]{\label{ta:#1}}
\newcommand{\TA}[1]{Table~\ref{ta:#1}}
\newcommand{\RTA}[1]{\ref{ta:#1}}

% List beginnings and endings
\newcommand{\bl}{\bss\begin{itemize}}
	\newcommand{\el}{\vspace{-.5\baselineskip}\end{itemize}\ess}
\newcommand{\ben}{\bss\begin{enumerate}}
	\newcommand{\een}{\vspace{-.5\baselineskip}\end{enumerate}\ess}

% Figure and table beginnings and endings
\newcommand{\bfg}{\begin{figure}}
	\newcommand{\efg}{\end{figure}}
\newcommand{\bt}{\begin{table}}
	\newcommand{\et}{\end{table}}

% Tabular and center beginnings and endings
\newcommand{\bc}{\begin{center}}
	\newcommand{\ec}{\end{center}}
\newcommand{\btb}{\begin{center}\begin{tabular}}
	\newcommand{\etb}{\end{tabular}\end{center}}

% Single space command
\newcommand{\bss}{\singlespacing}
\newcommand{\ess}{\doublespacing}

%---New environment "arbspace". (modeled after singlespace environment
%                                in Doublespace.sty)
%   The baselinestretch only takes effect at a size change, so do one.
\def\arbspace#1{\def\baselinestretch{#1}\@normalsize}
\def\endarbspace{}
\newcommand{\bas}{\begin{arbspace}}
	\newcommand{\eas}{\end{arbspace}}

% An explanation for a function
\newcommand{\explain}[1]{\mbox{\hspace{2em} #1}}

% Quick commands for symbols
\newcommand{\half}{\frac{1}{2}}
\newcommand{\third}{\frac{1}{3}}
\newcommand{\twothird}{\frac{2}{3}}
\newcommand{\mdot}{\dot{m}}
\newcommand{\ten}[1]{\times 10^{#1}\,}
\newcommand{\cL}{{\cal L}}
\newcommand{\cD}{{\cal D}}
\newcommand{\cF}{{\cal F}}
\newcommand{\cE}{{\cal E}}
\newcommand{\cS}{{\cal S}}
\newcommand{\mA}{\mathbf{A}}
\newcommand{\mX}{\mathbf{X}}
\newcommand{\mU}{\mathbf{U}}
\newcommand{\mW}{\mathbf{W}}
\newcommand{\mSigma}{\mathbf{\Sigma}}
\newcommand{\mS}{\mathbf{S}}
\newcommand{\mB}{\mathbf{B}}
\newcommand{\mC}{\mathbf{C}}
\newcommand{\mD}{\mathbf{D}}
\renewcommand{\Re}{\mbox{Re}}
\newcommand{\Ma}{\mbox{Ma}}

% Inclusion of Graphics Data
%\input{psfig}
%\psfiginit

% More Quick Commands
\newcommand{\bi}{\begin{itemize}}
	\newcommand{\ei}{\end{itemize}}
\newcommand{\dxi}{\Delta x_i}
\newcommand{\dyj}{\Delta y_j}
\newcommand{\ts}[1]{\textstyle #1}

\begin{document}

\title{Dynamic Mode Decomposition for Subcritical Metal Systems} %title of paper

% Use the \addAuthor macro to add authors in the order they should appear. The second argument corresponds to
% the affiliation declared below.
% The corresponding author should be wrapped in \correspondingAuthor
\addAuthor{Zachary K. Hardy}{a}
\addAuthor{\correspondingAuthor{Jim E. Morel}}{a}
% The corresponding author's email can be specified using \correspondingEmail
\correspondingEmail{morel@tamu.edu}
\addAuthor{Cory Ahrens}{b}

% Affiliations can be added in the order they should appear. For breaks in addresses, use either \\ or \tabularnewline
\addAffiliation{a}{Texas A\&M University, Department of Nuclear Engineering\\ 423 Spence St., MS 3133\\  AI Engineering Building\\ College Station, TX 77843-3133}
\addAffiliation{b}{Los Alamos National Laboratory\\ PO Box 1663\\ Los Alamos, NM 87545-1362}

% Add keywords to appear in Abstract in the order they should appear
\addKeyword{alpha-eigenvalues}
\addKeyword{dynamic mode decomposition}
\addKeyword{subcritical systems}

\titlePage

\begin{abstract}
In this paper we explore the use of Dynamic Mode Decomposition (DMD) for modeling the kinetics of subcritical metal systems pulsed with fast neutrons.  
Our ultimate purpose is to obtain a fast and accurate reduced-order model for such systems that can be used to develop an emulator.  
An alternative to DMD is $\alpha$-eigenfunction expansions, but we show that DMD is vastly superior in several ways for the systems of interest to us.  
\end{abstract}


\section{Introduction}
Modeling of subcritical metal systems subjected to short bursts of fast neutrons is challenging.  
While such modeling can certainly be done with a time-dependent transport code, we want to develop a reduced-order model that can provide physical insight into such systems and form the basis of an emulator in order to solve inverse problems quickly and compute sensitivities economically.  
One method of developing a reduced-order model is a truncated $\alpha$-eigenfunction expansion.  
To the authors' knowledge, no proof of the incompleteness of the alpha eigenfunctions exists.  
It seems, however, that because of the singular nature of the uncollided and first collided flux \cite{bondarenko1998structure} \cite{larsen1975solutions}, the $\alpha$-eigenfunctions likely do not form a complete basis.  
However, Jorgens proved, under rather general assumptions, that the $\alpha$-eigenfunctions form an ``asymptotically complete'' basis in the following sense: for times $t>3\tau$, where $\tau$ is a representative time for neutrons to stream across the spatial domain, $\psi\left(t\right) \sim \sum_{n=0}^{N-1} A_n \psi_n e^{\alpha_n t}$, where the remainder term is proportional to $e^{\alpha_N t}$ and the amplitudes $A_n$ depend on the initial condition \cite{jorgens1958asymptotic}.  
From this representation and the fact that the eigenvalues are ordered as $...Re\left(\alpha_3\right)\le Re\left(\alpha_2\right)\le  Re\left(\alpha_1\right) < \alpha_0$, one sees that $|\alpha_0 - \alpha_1|^{-1}$ sets a time scale for the higher-order modes to decay relative to the fundamental mode.  
The relationship of this time scale compared with the time over which the experiment takes place can be significantly different for subcritical compared to supercritical systems.  

Indeed, for a supercritical system, where $\alpha_0 > 0$ and the real part of $\alpha_n$, $n=1,2,3,...$ is negative, the transport solution is quickly dominated by the fundamental mode $\alpha$-eigenfunction.  
However, the situation can be different for a subcritical system.  
If the flux (experimental signal) decays much faster than $|\alpha_0 - \alpha_1|^{-1}$, then the fundamental mode might not contribution significanlty to the signal before it is negligibly small.  
The contribution of the fundamental mode depends upon the extent to which it is present in the expansion for the initial condition.  
In a subcritical system the fundamental mode will be dominated by the slowest neutrons, which will be thermal neutrons.  
In a metal system subjected to a burst of fast neutrons, there will be almost no thermal neutrons created before the response of the system has effectively died away.  
Thus it seems likely that an $\alpha$-eigenfunction expansion will be particularly inefficient for such systems, because a high degree of cancellation between the eigenfunctions will be required to achieve a negligible thermal neutron component in the initial condition.  

Dynamic Mode Decomposition (DMD) is a reduced order technique for modeling dynamical systems.  
It has been applied in many areas and is perhaps best known for its use in the fluid-flow community 
\cite{kutz2016dynamic} \cite{schmid2010dynamic} \cite{jovanovic2012low}.  
To the authors' knowledge the application of DMD to neutronics problems is limited \cite{abdo2018data} \cite{mcclarren2018calculating}.  
The purpose of this paper is to perform a preliminary investigation of DMD as an alternative to $\alpha$-eigenfunction expansions for modeling pulsed neutron experiments.  
For this initial study, we use a very approximate but relevant high-dimensional model consisting of a time-dependent 1-D spherical geometry three-group diffusion approximation.  
We have an analytic eigenfunction solution for these equations, as well as a computer code for solving these equations with second-order accuracy in both time and space.  
We later describe DMD in detail, but at this point it suffices to say that given a time series of vector ``snapshots'' from a simulation or experiment, DMD produces a time-dependent solution for those snapshots that is constructed from a sum of snapshot modes, each with an exponential decay rate.  
For instance, the snapshots could be space-dependent, three-group diffusion scalar fluxes from a time-dependent calculation.  
They could also be an analogous time series of any quantity of interest computed from the diffusion solution, such as space-dependent reaction rates.  
The form of the DMD solution is identical to that of an alpha-eigenfunction solution.  
However, the DMD modes can only contain what the snapshots contain, so if the snapshots lack a thermal neutron component, so will the DMD modes.  
This suggests that fewer DMD modes should be required for our problems than $\alpha$-eigenfunctions.  

Indeed we present results demonstrating that the DMD method is accurate and more efficient than $\alpha$-eigenfunction expansions for our problems.  
The remainder of this paper is organized as follows.  
First we describe the particular variant of the DMD method that we use.  
Then we describe the three-group diffusion model, followed by a description of our space-time discretization for the diffusion model.  
Finally, numerical results are given, followed by conclusions and recommendations for future work.  

\section{The DMD Method}
There are many variations of the DMD method.  
Here we describe the variant that we use to model subcritical fissioning systems \cite{kutz2016dynamic} \cite{schmid2010dynamic}.  
One begins with a time series of vector snapshots, $\{\vec{v}_j\}_{j=1}^{N}$, uniformly sampled in time, that are associated with the dynamical system.  
We assume that at least the first $N-1$ snapshots are linearly-independent.  
A process for ensuring this is later discussed.  
Each vector has $M$ real components with $M \gg N$.  
For example, each snapshot could represent a discrete angular flux solution in a time-dependent transport calculation, or it could represent some space-dependent quantity of interest associated with that solution.  
Note that the snapshots can be obtained either computationally or experimentally.  
For our application, we generate them computationally.  
It is further assumed that there exists a temporal matrix $\mathbf{A}$, that maps each snapshot to its successor:
\be
	\vec{v}_{i+1} = \mA \vec{v}_{i} \pec \quad i=1,N-1.
\LEQ{2.1}
\ee

Let us consider the subspace of $M$-vectors, which we denote by $\cS$, spanned by the first $N-1$ snapshots.  
If $\vec{v}_N$ were replaced by a least-squares fit from $\cS$, then $\mA$ would be uniquely defined as a mapping from $\cS$ to $\cS$, and be represented by the following matrix with respect to the snapshot basis:
\be
	\mA_s = \bracket{
		\begin{array}{ccccccc}
			0 & 0 & ....& 0 & c_1 \\
			1 & 0 & ....& 0 & c_2 \\
			0 & 1 & ....& 0 & c_3 \\
			. & . & ....& . & . \\
			. & . & ....& . & . \\
			. & . & ....& . & . \\
			0 & 0 & ....& 0 & . \\
			0 & 0 & ....& 1 & c_{N-1}
	\end{array}
	} \pec
	\LEQ{2.2}
\ee
where the least-squares fit to $\vec{v}_N$ from $\cS$ is represented by
\be
	\widehat{v}_N = \sum_{j=1}^{N-1} c_j \vec{v}_j \pep
	\LEQ{2.3}
\ee
For example, in the snapshot basis,
\be
	\vec{v}_1 = (1,0,0,0,...0)^T \pec
	\LEQ{2.4}
\ee
and 
\be
	\mA \vec{v}_1 = \vec{v}_2 = (0,1,0,0,...0)^T \pec
	\LEQ{2.5}
\ee
Note that if $c_1$ is non-zero, $\mA$ will be invertible on $\cS$, and thus will have $N-1$ non-zero eigenvalues.  
We will define $\mA$ in this manner with respect to $\cS$, assume that it is invertible on $\cS$, and further define all of its remaining eigenvalues to be zero.  
 
Let $\{\lambda_i\}_{i=1}^{N-1}$ denote the non-zero eigenvalues of $\mA$, and let $\{\vec{z}_i\}_{i=1}^{N-1}$ denote the corresponding eigenvectors.  
Then the dynamic solution is given by 
\be
	\vec{v}(t) = \sum_{i=1}^{N-1} a_i \vec{z}_i \exp{(\omega_i t)} \pec
\LEQ{2.5a}
\ee
where

\be
	\omega_i = \frac{\ln{(\lambda_i)}}{\Delta t} \pec
	\LEQ{2.5b}
\ee
where $\Delta t$ is the time between snapshots and the expansion coefficients, 
$\{a_i\}_{i=1}^{N-1}$ are determined by the initial condition:
\be
	\vec{v}(0) = \vec{v}_1 = \sum_{i=1}^{N-1} a_i \vec{z}_i \pep
	\LEQ{2.5c}
\ee
The dynamic solution exactly reproduces each of the snapshots at its time of 
sampling with the exception of the last snapshot, which is approximated with 
its least-squares fit.  
This follows from the fact that at the end of the first sampling period, $\mA$ 
is effectively applied to $\vec{v}_1$, and reapplied at the end of each 
sampling period thereafter.  
For instance, 
\bea
	\vec{v}(\Delta t) &=& \sum_{i=1}^{N-1} a_i \vec{z}_i \exp{(\omega_i \Delta t)} \pec \nonumber \\
	&=& \sum_{i=1}^{N-1} a_i \vec{z}_i \lambda_i \pec \nonumber \\ 
	&=& \mA \vec{v}_1 \pec \nonumber \\ 
	&=& \vec{v}_2 \pep
	\LEQ{2.5d}
\eea 
Note that any eigenfunction with a zero-eigenvalue will have no dynamic content because it will instantly attenuate to zero.  
Thus if $\widehat{v}_N$ has a zero value for $c_1$ in \EQ{2.3}, $\mA$ will have only $N-2$ non-zero eigenvalues, and one can simply ignore the eigenfunction associated with the extra zero eigenvalue.  
We henceforth continue to assume that $\mA$ has $N-1$ non-zero eigenvalues for simplicity, but without loss of generality.  

We now describe a process that enables us to compute the non-zero eigenvalues and the eigenvectors of $\mA$.  
Furthermore, we need not explicitly compute $\widehat{v}_N$.  
First we express \EQ{2.1} as follows, neglecting the substitution of $\widehat{v}_N$ for $\vec{v}_N$ for reasons later explained.  
\be
	\mX_2^{N} = \mA \mX_{1}^{N-1} \pec
\LEQ{2.6}
\ee
where $\mX_2^{N}$ is the $M \times N-1$ matrix whose colunms are the vectors $\vec{v}_2$ through $\vec{v}_{N}$, and $\mX_{1}^{N-1}$ is the $M\times N-1$ matrix whose colunms are the vectors $\vec{v}_1$ through $\vec{v}_{N-1}$.  

We next perform a singular value decomposition (SVD) of $\mX_{1}^{N-1}$:
\be
	\mX_{1}^{N-1} = \mU \mSigma \mW^T \pec
\LEQ{2.7}
\ee
where $\mU$ is a $M \times M$ orthogonal matrix, $\mSigma$ is a $M\times N-1$ matrix with $N-1$ non-zero singular values on the diagonal and zeros everywhere else, and $\mW^T$ is a $N-1 \times N-1$ orthogonal matrix.  
Substituting from \EQ{2.6} into \EQ{2.7}, we obtain, 
\be
	\mX_2^{N} = \mA \mU \mSigma \mW^T \pep
\LEQ{2.8}
\ee
Next we multiply \EQ{2.8} on the left by $\mU^{T}$ and on the right by
$\mW\mSigma^{-1}$ to obtain 
\be
	\mS = \mU^{T}\mX_2^{N}\mW\mSigma^{-1} = \mU^{T}\mA\mU \pec
	\LEQ{2.10}
\ee
where $\mSigma^{-1}$ is the pseudo inverse of $\mSigma$.  
The pseudo-inverse is obtained from $\mSigma$ simply by first inverting its non-zero elements and then transposing it.  
Note that the right side of \EQ{2.10} is a $M \times M$ matrix that represents a similarity transformation of $\mA$, and thus $\mS$ has the same eigenvalues as $\mA$.  
Furthermore, in accordance with the similarity transformation, if $\vec{y}$ is an eigenvector of $\mS$, then $\vec{z} = \mU \vec{y}$ is an eigenvector of $\mA$.  

The structure of $\mS$ is important because it enables us to avoid computing the entire $M \times M$ matrix together with its eigenvalues and eigenvectors.  
More specifically, all the columns of $\mS$ beyond column $N-1$ are zero, and we represent the remainder of the matrix as an $N-1 \times N-1$ matrix, $\mS_{T}$, in the upper top left corner and a $M-N+1 \times N-1$ matrix, $\mS_B$, in the bottom left corner:
\be
	\mS = \bracket{
		\begin{array}{cc}
			\mS_T  & 0 \\
			\mS_B & 0  
		\end{array}
	} \pep
	\LEQ{2.11}
\ee
It can be shown that the eigenvalues of $\mS_T$ are the non-zero eigenvalues of $\mS$, and hence, the non-zero eigenvalues of $\mA$.  
Furthermore, there is a very simple way to compute the corresponding eigenvectors of $\mS$ from the eigenvectors of $\mS_T$.  
However, it is unnecessary to compute any eigenvectors other than those of $\mS_T$.  
To explain why this is so, we first note that the there is no difference between the eigenvectors of $\mS_T$ and the vectors formed by the first $N-1$ components of the corresponding eigenvectors of $\mS$.  
Thus if one computes the eigenvalues and eigenvectors of $\mS_T$, one has the non-zero eigenvalues of $\mA$ and $\mS$, and the corresponding eigenvectors of $\mS$ truncated to a length of $N-1$.

The first step in the calculation of the non-zero eigenvalues and corresponding 
eigenvectors of $\mA$ is to directly compute $\mS_T$ as follows:
\be
	\mS_T = \mU_{N-1}^{T} \mX_{2}^{N}\mW\mSigma^{-1}_{N-1} \pec
	\LEQ{2.12}
\ee
where $\mU_{N-1}$ is the $M \times N-1$ matrix composed of the first $N-1$ columns of $\mU$, and $\mSigma_{N-1}$ is the $N-1 \times N-1$ matrix formed by the first $N-1$ rows of $\mSigma$.  

At this point we have sufficient information to explain why one need not substitute $\widehat{v}_N$ for $\vec{v}_N$ in $\mX_2^{N}$.  
To this end we express the equations for the expansion coefficients of the least-squares fit to $\vec{v}_N$ as follows:
\be
	\mU^T_{N-1}\fn{\widehat{v}_N - \vec{v}_N} = \vec{0} \pec
	\LEQ{2.9}
\ee
where $\mU_{N-1}$ is the $M \times N-1$ matrix consisting of the first $N-1$ columns of $\mU$.  
Note that by virtue of the SVD of $\mX_{1}^{N-1}$, these column vectors represent an orthonormal basis for the first $N-1$ snapshots.  
Thus $\widehat{v}_N$ does indeed represent a least-squares fit to $\vec{v}_N$ from $\cS$.  
From \EQ{2.9} it follows that the first $N-1$ components of $\mU^T \vec{v}_N$ are identical to those of $\mU^T \widehat{v}_N$.  
Thus it can be seen from \EQ{2.12} that $\mS_T$ is invariant to the substitution of $\widehat{v}_N$ for $\vec{v}_N$ in $X_{2}^{N}$.  

Finally, we compute the non-zero eigenvectors of $\mA$ simply by multiplying the eigenvectors of $\mS_T$ by $U_{N-1}$.  
\be
	\vec{z}_i = \mU_{N-1} \vec{x}_i \pec \quad i=1,N-1,
	\LEQ{2.13}
\ee
where $\vec{z}_i$ and $\vec{x}_i$ are the $i$'th eigenvectors of $\mA$ and $\mS_T$, respectively.  
This is clearly much more economical than computing the eigenvectors of $\mS$ and multiplying them by $\mU$.  
This simplification is justified by the fact that since the first $N-1$ columns of $\mU$ form an orthonormal basis for $\cS$, the non-zero eigenvectors of $\mA$ must lie in this space.  
Thus columns of $\mU$ beyond $N-1$, or equivalently, elements of the eigenvectors of $\mS$ below the $N-1$ position do not contribute to the non-zero eigenvectors of $\mA$.  

As previously discussed, the SVD of $\mX_{1}^{N-1}$ can be used to determine if the first $N-1$ snapshots are sufficiently linearly independent by inspection  of the singular values.  
Below some tolerance, one can consider the singular values to be zero, and discard the corresponding information in the various component matrices.  
More specifically, if we assume that only the first $K$ singular values are considered non-zero, then in \EQ{2.12} only the first $K$ columns of $\mU_{N-1}$ and $\mX_{2}^{N}$ are retained, and only the first $K$ columns and first $K$ rows of $\mSigma_{N-1}$ and $\mW^T$ are retained.  
We currently set any singular value that is less than $10^{-8}$ times the largest singular value to zero. 
It is not generally clear as to what tolerance is the most efficient.  
We determine this on a case-by-case basis.  

\section{The Three-Group Diffusion Model}
For simplicity, a three group diffusion model in a bare homogeneous fissioning sphere with an atom density of $N=0.05$ atoms/cc is considered.  
The nuclear data is tabulated in \TA{xs_data}, where $\chi$ is the fission spectrum, $\nu\sigma_f$ is the fission neutron production cross section, $\sigma_R$ is the removal cross section, including absorption and outscattering, $\sigma_t$ is the total cross section, and $v$ is the velocity.  
As is apparent from the table, fission neutrons are only born into the fast group indicating that the only production of epithermal and thermal neutrons comes from downscattering.  
The scattering matrix is shown in \TA{scat}.  
There is only scattering from the fast to epithermal group and no upscattering 
whatsoever.  

With this model it is assured that thermal neutrons only appear in the problem if they exist in the initial condition.  
Using an appropriate initial condition, the characterization of the performance of DMD becomes quite unambiguous; in a problem where thermal neutrons are, in fact, irrelevant to the solution, the DMD solution should not have a thermal component.  
\bt[h] \centering 
	\caption{Nuclear data for the three-group model.} 
	\btb{|c|c|c|c|}
		\hline
		\diagbox{Reaction}{Group} & Fast  & Epithermal  & Thermal  \\  \hline
		$\chi$  & 1 & 0 & 0 \\ 	\hline
		$\nu\sigma_f \ [b]$ & 5.4 & 60.8 & 28.0 \\  \hline
		$\sigma_R \ [b]$  & 3.31 & 36.2 & 13.6 \\  \hline
		$\sigma_t \ [b]$ & 7.71 & 50.0 & 25.6\\ \hline
		$v \ [cm/\mu s]$ & 2000 & 100 & 2.2 \\  \hline
	\etb 
	 \LTA{xs_data}
\et

\bt[h] \centering 
	\caption{Scattering matrix given in barns.} 
	\btb{|c|c|c|c|}
		\hline
		\diagbox{From}{To}& Fast  & Epithermal  & Thermal  \\  \hline
		Fast  & 0 & 1.46 & 0 \\  \hline
		Epithermal & 0 & 0 & 0 \\  \hline
		Thermal  & 0 & 0 & 0 \\  \hline
	\etb 
	\LTA{scat}
\et

This system is governed by the time-dependent multi-group neutron diffusion equations.  
The center of the spherical system is reflective and a zero-flux condition is imposed at the physical boundary.  
The initial condition will be parabolic in space for both the fast and epithermal groups and zero for the thermal group.  
This is given in \EQ{mg_diff},
\be
	\begin{cases}
		\dxdy{\phi_g}{t} - \nabla \cdot D_g \nabla \phi_g + \Sigma_{R,g} \phi_g = \chi_g \sum\limits_{g'=1}^{3} 
			\nu\Sigma_{f,g'} \phi_{g'} + \sum\limits_{g'=1, \ g' \neq g}^g \Sigma_s^{g' \rightarrow g} \phi_{g'} \\
		-D_g \dxdy{\phi_g}{r} \Big\rvert_{(0,t)} = 0, \pev \phi_g(R,t) = 0 \\
		\phi_1(r,0) = \phi_2(r,0) = \frac{R^2 - r^2}{R^2}, \pev \phi_3(r,0) = 0
	\end{cases} ,
	\LEQ{mg_diff}
\ee 
where $\Sigma$'s are macroscopic cross sections, given by $\Sigma = N\sigma$, and $D_g$ is the diffusion coefficient, given by $1/3\Sigma_t$.  
This can be simplified based on the nuclear data to \EQ{simp_diff}.
\be
	\begin{cases}
		\dxdy{\phi_g}{t} - \nabla \cdot D_g \nabla \phi_g + \Sigma_{R,g} \phi_g = \chi_1 \sum\limits_{g'=1}^{3} 
			\nu\Sigma_{f,g'} \phi_{g'} + \Sigma_s^{1 \rightarrow 2} \phi_{1} \\
		-D_g \dxdy{\phi_g}{r} \Big\rvert_{(0,t)} = 0, \pev \phi_g(R,t) = 0 \\
		\phi_1(r,0) = \phi_2(r,0) = \frac{R^2 - r^2}{R^2}, \pev \phi_3(r,0) = 0.
	\end{cases} 
	\LEQ{simp_diff}
\ee
This problem can be solved using an doubly-indexed $\alpha$-eigenfunction expansion.  
The expansion takes the form shown in \EQ{exp},
\be
	\phi_g(r, t) = \sum_{n, m = 1}^{\infty, 3} A_{nm} \varphi_n(r) f_{nm, g} e^{\alpha_{nm} t},
	\LEQ{exp} 
\ee
where $n$ gives the index of the spatial mode and $m$ gives the index of the energy modes corresponding to each spatial mode.  
$A_{nm}$ is the coeffient for the $n,m$'th space-energy mode, $\varphi_n(r)$ is 
the $n$'th spatial mode, $f_{nm, g}$ is the $g$'th component of the $n,m$'th space-energy mode, and $\alpha_{nm}$ is the time eigenvalue for the $n,m$'th space-energy mode.  
The spatial modes are those of the Laplace equation for a spherical geometry given by \EQ{eigfunc}, with the same boundary conditions as those in the governing equation.  
\be
	\varphi_n(r) = \frac{1}{r} \sin\fn{ \frac{n \pi r}{R} }
	\LEQ{eigfunc}
\ee
For simplicity, the buckling approximation is used in place of the Laplacian, as shown in \EQ{buckle}.  
\bea 
	\begin{aligned}
		- \nabla^2 \varphi_n(r) &= B_n^2 \varphi_n(r) \\
		B_n^2 &= \fn{ \frac{n \pi}{R} }^2
	\end{aligned} 
\LEQ{buckle} 
\eea
Plugging in the eigenfunction expansion into the governing equation, rearranging, and taking advantage of the orthogonality of the spatial eigenfunctions, an eigenvalue problem is formed for each spatial mode, given by \EQ{eigprob}.  
\be
	\alpha_{nm} f_{nm, g} = v_g \fn{ - \fn{ D_g B_n^2 + \Sigma_{R,g} }f_{nm, g} + \chi_1 \sum_{g' = 1}^{3} 
		\nu\Sigma_{f,g'} f_{nm, g'} + \Sigma_s^{1 \rightarrow 2} f_{nm, 1} } 
	\LEQ{eigprob}
\ee
It should be clear that this forms a 3 $\times$ 3 system which will yield 3 distinctive $\alpha$-eigenvalues with corresponding eigenvectors of length 3 for each spatial mode used in the expansion.

The coefficients for each space-energy mode is computed by a projection of the initial condition onto each respective space-energy mode.  
Because the eigenproblem described above is not symmetric, the adjoint (left) eigenfunctions given by, $f_{nm}^* \varphi_n(r)$, must be used.  
The spatial eigenfunctions do not have the adjoint designation because they are self-adjoint.  
The coefficients are computed according to \EQ{coeff},
\be
	A_{nm} = \frac{\ave{ f_{nm}^* \varphi_n(r), \ \phi(r, 0)}}{\ave{f_{nm}^* \varphi_n(r), \ f_{nm} 
		\varphi_n(r)}},
	\LEQ{coeff} 
\ee
where $\ave{\cdot}$ designates the inner product, an integration and sum, over space and energy, respectively.  
To generate the full analytic solution modes are generated and added to the solution until the $L^2$-fit to the initial condition changes by less than a specified tolerance.  
This expansion is then propegated forward in time with the $\alpha$-eigenfunctions to some specified time.  

\section{Discretization of the Diffusion Equations}
In practice,  $\alpha$-eigenvalue calculations for this application are quite computationally expensive, and further, in most cases, analytic solutions are not available.  
For this reason, it is desireable to develop a numerical method so that the accuracy of DMD can be characterized in a realistic way and compared against a known solution.  
This section outlines the development of a second-order accurate method using cell-centered finite volume discretization in space and trapezoidal second-order backwards difference (TBDF-2) discretization in time \cite{edwards2011nonlinear}.  

In the spatial discretization, the domain is broken into $N$ cells, with cell width $h$, where the cell centers carry integer indiced and the edges half-integer indices.  
Consider \EQ{cell_eqn}, the conservation equation for the i$^{\text{th}}$ cell,
\be
	\frac{1}{v_g} \dxdt{\phi_{i,g}} - \nabla D_{g} \nabla \phi_{i,g} + 	
	\Sigma_{R,g} \phi_{i,g}  =  S_{i,g},
	\LEQ{cell_eqn} 
\ee
where $S_{i,g}$ contains the production terms for group $g$ neutrons in cell $i$.  
The reader should note that source term is simply the fission and scattering source.  
Integrating this equation over the volume of the i$^{\text{th}}$ cell, one obtains:
\be
	\frac{V_i}{v_g} \dxdt{\phi_{i,g}} - A_{i+\half} J_{i+\half,g} + A_{i-\half} 
	J_{i-\half,g} + V_i \Sigma_{R,g} \phi_{i,g} = V_i S_{i,g}
\ee
where $V_i $ is the volume of cell $i$, $ A_{i+1/2} $ is the surface area of the outer edge of cell $i$, and $ J_{i+1/2} = D_g \nabla \phi_{g,i} $ is the current at outer edge of cell $i$.  
Central difference is now applied to the current terms with respect to adjacent cell-centers to obtain the second-order accurate scheme, as shown in \EQ{space_discretized}.  
\be
	\frac{V_i}{v_g} \dxdt{\phi_{i,g}} - A_{i+\half} \frac{\phi_{i+1,g} - \phi_{i,g}}{h} + A_{i-\half} \frac{\phi_{i,g} 
		- \phi_{i-1,g}}{h} + V_i \Sigma_{R,g} \phi_{i,g} = V_i S_{i,g},
	\LEQ{space_discretized} 
\ee

The TBDF-2 temporal discretization is advantageous because it is second-order accurate and strongly damps fast time-scale components of the solution, unlike methods such as Crank-Nicholson, which can be highly oscillatory for stiff systems.  
This method first takes half of a time-step using Crank-Nicholson, and is shown for an arbitrary linear system in \EQ{halfCN}.  
\be
	\frac{2 \fn{ f^{n+\half} - f^n }}{\Delta t} =  \half A \fn{f^{n+\half} + f^n}.
	\LEQ{halfCN} 
\ee
The result of solving for $f^{n+\half}$ is then plugged into second-order backwards difference to obtain the solution at the end of the time step, shown in \EQ{halfBDF}.  
\be
	\frac{3 \fn{ f^{n+1} - f^{n+\half} }}{\Delta t} - \frac{f^{n+\half} - 
		f^n}{\Delta t} = A f^{n+1}.
	\LEQ{halfBDF} 
\ee 
Implementing this is relatively straightforward. 
The time derivative term containing the scalar flux at ther next half-time step gets added into the operator and the terms containing the scalar flux from previous time-steps become a source term.  
Overall, this method is second order accurate and was verified against the analytic solution.  
\FI{2ord} shows the error as a function of number of spatiotemporal grid halvings, where a slope of 2 is observed.
\bfg[h] \centering
	\includegraphics[scale=0.5]{figures/NSE19-11Fig01.jpg}
	\caption{Convergence of the numerical solution to the analytical}
	\LFI{2ord}
\efg

\section{Numerical Results}
In this section numerical results are presented which characterize the efficiency and accuracy of DMD for this application using both the scalar group flux and fission rate.  
The DMD solution presented in this section is derived from data produced by the numerical model discussed above for a 5 cm sphere over a 10 ns simulation time discretized into 500 spatial cells and 30 time-steps.  
Separate sections will describe the application of DMD on both the scalar group flux and the fission rate.  
Each section will present the singular value spectrum for the simulation data, show the convergence of the DMD solution to the simulation data in the Frobenius norm, show a few snapshots of the DMD solution overlayed on the exact solution, and give results for the interpolative and extrapolative abilities of DMD between snapshots and beyond the data collection period, respectively.  
In addition, the performance of DMD on the scalar group flux will and the physical insights gained will be compared with that of an $\alpha$-eigenfunction computation.  

\subsection{DMD with the Scalar Group Flux}
In order for DMD to be a viable reduced order model, a low-rank structure must exist within the data.  
Because singular values are related to the eigenvalues matrix it decomposes, the singular values of the rectangular matrix $\bm{X}^{N-1}_1$ contains information about the dominant time-eigenvalues during the duration of the simulation.  
The singular value spectrum of  $\bm{X}^{N-1}_1$ is presented below in \FI{sv-spec}, and shows that this condition is met.  

\bfg[h] \centering
	\includegraphics[scale=0.5]{figures/NSE19-11Fig02.jpg}
	\caption{Singlar value spectrum of numerical snapshot series}
	\LFI{sv-spec}
\efg

When implementing the DMD method, one often may want to use fewer modes than the $N-1$, the maximum number of modes obtained without any truncation.  
Because very small singular values correspond to very little dynamics information, it becomes important to understand how to choose the number of dynamic modes to keep.  
This can be done by specifying a relative singular value cutoff such that all singular values smaller are set to zero and neglected, or by specifying the number of modes desired.  
If the snapshot series is denoted as $X$, and the DMD representation of $X$, $X_{DMD}$, theory states that the error in the Frobenius norm should be proportional to the magnitude of the largest truncated singular value, given by $\lVert X - X_{DMD} \rVert_F \sim \sigma_{K+1}$ \cite{kutz2016dynamic}.  
This is code verification is demonstrated in \FI{recovery}.  
\bfg[h] \centering
	$
	\begin{array}{cc}
		\includegraphics[scale=0.5]{figures/NSE19-11Fig03a.jpg} &
		\includegraphics[scale=0.5]{figures/NSE19-11Fig03b.jpg}
	\end{array}
	$
	\caption{DMD solution recovery study}
	\LFI{recovery}
\efg
Practicallty, only the number of modes corresponding to error on the order of the simulation error need be kept.  
For example, if the numerical simulation has error on the order of $10^{-4}$, from the singular value spectrum above, only 5 dynamic modes are needed to maintain the same level of accuracy to the analytic solution.  
For this reason, it is recommended that a singular value cutoff is used, and set, to satisfy this criteria. 

To present a visual representation of DMD's ability to reproduce the solution it decomposes, the DMD solution and exact solutions are plotted against each other at 1/3, 1/2, and 2/3 simulation time, respectively in \FI{flx-plt}.  
For this case, the numerical solution is accurate to the exact solution to 4 digits in an $L^2$ sense.  
This plot qualitatively shows both the accuracy of the DMD solution throughout the duration of the simulation and its accuracy at times corresponding to a snapshot and times in between snapshots.  
\bfg[t] \centering
	$
	\begin{array}{cc}
		\includegraphics[scale=0.45]{figures/NSE19-11Fig04a.jpg} &
		\includegraphics[scale=0.45]{figures/NSE19-11Fig04b.jpg} \\
	\end{array}
	$
	\includegraphics[scale=0.45]{figures/NSE19-11Fig04c.jpg}
	\caption{Scalar group flux at 1/3, 1/2, and 2/3 simulation time}
	\LFI{flx-plt}
\efg

More quantitatively, the interpolation error can be seen in \FI{interp-flux}.  
In this plot, the DMD and analytic solutions are sampled on a more refined time grid to obtain errors both on and in between the snapshots used to build the DMD solution.  
The DMD solution for the scalar group flux clearly does not lose accuracy when interpolating between snapshots.  
\bfg[!htb] \centering
	\includegraphics[scale=0.5]{figures/NSE19-11Fig05.jpg}
	\caption{Scalar group flux DMD interpolation error}
	\LFI{interp-flux}
\efg
Another property worth investigating is DMD's ability to extrapolate to times beyond the data collection period.  
These results are presented in \FI{extrap-flux}.  
\bfg[!htb] \centering
	\includegraphics[scale=0.5]{figures/NSE19-11Fig06.jpg}
	\caption{Scalar group flux DMD extrapolation error}
	\LFI{extrap-flux}
\efg
This simulation was carried out to 5 times the data collection period using the same time-step size as the original simulation.  
It is clear that DMD models this system to a similar degree of accuracy even long after the sampling period.  
For this case specifically, as time passes, all higher order modes become less relevant to the solution.  
These results suggest that the DMD solution captured the dynamics of the longer lived modes such that as the system settles into those modes, DMD does as well.  
More information related to this is presented during the comparison of the dynamic modes to the $\alpha$-eigenfunctions.  

With the DMD code verified, and its capabilities thoroughly investigated, it is now important to characterize its performance against the traditional method used, an $\alpha$-eigenfunction expansion.  
The most useful way to draw this comparison is to investigate the error of each method to the exact solution as a function of number of modes.  
For the $\alpha$-eigenfunction expansion, modes from the exact solution are simply added in decreasing numerical order to ``emulate'' the convergence in an eigenvalue solver.  
To obtain the DMD-results, the SVD of the numerical data was simply truncated to the appropriate rank, for the respective datapoint.  
A comparison of each method is shown in \FI{comp}.  
\bfg[!htb] \centering
\includegraphics[scale=0.5]{figures/NSE19-11Fig07.jpg}
	\caption{Comparison of DMD to $\alpha$-eigenfunction expansions}
	\LFI{comp}
\efg

It is clear that DMD vastly outperforms the method of $\alpha$-eigenfunction expansions.  
Because eigenvalue solvers converge to eigenvalues in numerical order, many of the modes converged to initially are thermally dominated and have negligible contribution to the solution.  
This is what leads to the flat regions of the exact solution curve.  
As is apparent, DMD rapidly converges to the error of the simulation to the exact solution in fewer than 10 modes.  
This is because the modes are guaranteed to be relevant to the solution over the simulation time, unlike $\alpha$-eigenfunctions.  
If one were to increase the resolution in the numerical simulation, the minimum error in the DMD solution would decrease, giving it an even greater efficiency when compared to an $\alpha$-eigenvalue solver.  

To demonstrate the ability of DMD to extract relevant structures and produce physical insights, the structures produced from DMD will be compared with those from the $\alpha$-eigenfunction expansion.  
Because no thermals exist in the system, the computed amplitudes of the thermal modes are zero, however, they must still be computed before any relevant modes are obtained in a standard $\alpha$-eigenvalue solver.  
The first 3 thermal modes are shown below in \FI{thermal-modes}.  
These are compared to the first 3 dynamic modes to highlight the ability of DMD to produce relevant modes to the system - modes which contain no thermal neutrons.  
The $\alpha$ modes are denoted as such, and the DMD modes denoted with $\omega$.  
Each mode is normalized to its $L^2$ norm, so that an accurate comparison can be made.  
\bfg[!htb] \centering
	$
	\begin{array}{c c}
		\includegraphics[scale=0.5]{figures/NSE19-11Fig08a.jpg} &
		\includegraphics[scale=0.5]{figures/NSE19-11Fig08b.jpg} \\
		\includegraphics[scale=0.5]{figures/NSE19-11Fig08c.jpg} &
		\includegraphics[scale=0.5]{figures/NSE19-11Fig08d.jpg} \\
		\includegraphics[scale=0.5]{figures/NSE19-11Fig08e.jpg} &
		\includegraphics[scale=0.5]{figures/NSE19-11Fig08f.jpg}
	\end{array}
	$
	\caption{Comparison of first 3 thermal and dynamic modes.}
	\LFI{thermal-modes}
\efg

The exact solution is composed of many structures which describe the physics of the underlying system.  
For this reason, it is expected that the DMD modes bear so resemblence to them in shape and time-scale.  
To show this, the eigenvalues and eigenfunctions of the first 3 epithermal modes of the $\alpha$-eigenfunction expansion and dynamic modes are compared in \FI{modes}.  
Because no thermal modes contribute to the system, the epithermal modes, set the decay rates for this problem.  
\bfg[!htb] \centering
	$
	\begin{array}{c c}
		\includegraphics[scale=0.5]{figures/NSE19-11Fig09a.jpg} &
		\includegraphics[scale=0.5]{figures/NSE19-11Fig09b.jpg} \\
		\includegraphics[scale=0.5]{figures/NSE19-11Fig09c.jpg} &
		\includegraphics[scale=0.5]{figures/NSE19-11Fig09d.jpg} \\
		\includegraphics[scale=0.5]{figures/NSE19-11Fig09e.jpg} &
		\includegraphics[scale=0.5]{figures/NSE19-11Fig09f.jpg}
	\end{array}
	$
	\caption{Comparision of $\alpha$-eigenfunctions to dynamic modes}
	\LFI{modes}
\efg
In contrast to the previous comparison, the fundamental epithermal mode and the first dynamic mode are visibly indistinguishable, and have identical decay constants.  
The shapes of the second mode appear to be nearly identical, and the DMD decay constant is slightly different.  
In the third mode, the general shape is similar but there are wide deviations in the magnitudes of each component, respectively, and again, the eigenvalue further drifts away from that of the $\alpha$-eigenfunction.  
Because only a finite number of dynamic modes represent the system, it is not reasonable to expect DMD to reproduce more than a few modes accurately because the dynamic modes achieve the same level of accuracy as a much larger number of the $\alpha$-eigenmodes.  
In fact, if the simulation is run for a long time relative to the decay rate of the fundamental mode, the dynamic mode will converge to only the first, or first few $\alpha$-eigenfunctions.  

These results give a clear indication of the applicability of DMD to these applications.  
It not only showed greater efficiency than an $\alpha$-eigenvalue computation, but also preserved the physical insights which can be gained.  
Further, the DMD solution produced accurate results both at snapshot sampling times and in between them, as well as far after the sampling period.  

\subsection{DMD with the Fission Rate}
For real applications, the size of solution vector may become restrictively large.  
For example, if one considers neutron transport, rather than diffusion, increases the energy resolution,  performs a 2- or 3-D computation, or any combination, the solution vector size would become exponentially larger.  
For this reason, it would be  advantageous to apply DMD on some other lower dimensionsal quantity of interest, such as space-dependent reaction rates.  
For this particular application, the fission rate is what primarily governs the measurements of interest.  
What makes this inquiry particularly interesting is the fact that no analytical solution exists with the same form as a DMD solution, unlike the previous example.  
The following will present the results from the application of DMD onto the fission rate computed from the scalar group flux simulation data.  

As before, the singular-value spectrum should be the first check to determine whether a low-rank structure exists within a given data-set.  
This is shown in \FI{sv-fis}.
\bfg[!htb] \centering
	\includegraphics[scale=0.5]{figures/NSE19-11Fig10.jpg}
	\caption{Singular value spectrum for the fission rate}
	\LFI{sv-fis}
\efg
Similarly to that for the scalar group flux, the singular value spectrum, indicates that a low-rank structure exists within the space-dependent fission rate snapshot series.  
The error in the Frobenius norm of the DMD solution to the data which produced it converges to machine precision as theory dictates.  
This is shown in \FI{conv-fis}, and shows that DMD can decompose the fission rate as accurately as it did the scalar group flux.  
\bfg[!htb] \centering
	$
	\begin{array}{c c}
		\includegraphics[scale=0.5]{figures/NSE19-11Fig11a.jpg} &
		\includegraphics[scale=0.5]{figures/NSE19-11Fig11b.jpg}
	\end{array}
	$
	\caption{Fission rate DMD convergence}
	\LFI{conv-fis}
\efg

For visualization purposes, the DMD and analytic fission rates are plotted against each other at 3 distinctive times, 1/3, 1/2, and 2/3 of the simulation time, to depict the solution at times both on an in between times corresponding to the snapshots used in DMD.  
This is shown in \FI{fis-ex}.
\bfg[!htb] \centering
	$
	\begin{array}{c c}
		\includegraphics[scale=0.5]{figures/NSE19-11Fig12a.jpg} &
		\includegraphics[scale=0.5]{figures/NSE19-11Fig12b.jpg}
	\end{array}
	$
	\includegraphics[scale=0.5]{figures/NSE19-11Fig12c.jpg}
	\caption{Fission rate at 1/3, 1/2. and 2/3 simulation time}
	\LFI{fis-ex}
\efg

The interpolative capabilities are explored in \FI{interp-fis}, and analogous results are obtained.
\bfg[!htb] \centering
	\includegraphics[scale=0.5]{figures/NSE19-11Fig13.jpg}
	\caption{Fission rate DMD interpolation error}
	\LFI{interp-fis}
\efg
Lastly, the extrapolative capabilities are explored in \FI{extrap-fis}. 
\bfg[t] \centering
	\includegraphics[scale=0.5]{figures/NSE19-11Fig14.jpg}
	\caption{Fission rate DMD extapolation error}
	\LFI{extrap-fis}
\efg
This obviously is a very different picture than that from the scalar group flux extrapolation error plot.  
The reason for this is believed to be because the fission rate does not have a solution of the same form as the DMD solution, and therefore, as was the case before, the long-term behaviors cannot be sufficiently captured.  
The reader should take note that the increase in error only increased by 2 orders of magnitude after 4 times the sampling period.  


\section{Conclusions and Recommendations for Future Work}
An initial study to characterize the perfomance of DMD on fast, subcritical, metal systems was performed on both the scalar group flux and the fission rate and to compare it to a truncated $\alpha$-eigenfunction expansion.  
It was shown that not only can DMD be applied to these problems successfully, but that DMD shows far superior efficiency to other methods while retaining the same level of physical insight.  
Unlike an $\alpha$-eigenfunction expansion, where many thermal modes with zero-amplitudes must be computed before contibuting modes, dynamic modes, because they are data driven, are guaranteed to be relevant to the solution.  
In fact, no thermal neutrons are present in any dynamic modes.  
The continuous DMD solution did not lose accuracy when interpolating between snapshots, and proved to maintain accuracy beyong the last sampling time.  
It should be stressed that the extrapolative capabilites of DMD are expected to be problem dependent.  

The ultimate goal for this work is to develop an emulator that uses DMD solutions evaluated at various points in a physical parameter space to accurately and efficiently interpolate within that parameter space.  
Much work needs to be done to in this regard.  
In the future, this method will be expanded to non-linear and multiphysics problems, to more accurately reflect the problems of interest.  
The prior is anticipated to have success because of the robust application of DMD to non-linear systems in the turbulent fluid flow community \cite{kutz2016dynamic} \cite{schmid2010dynamic}.  
Lastly, an exploration of different flavors of the DMD algorithm should be explored in order to find the best tailored algorithm for problems within this application space would be a useful endeavor.  

\section*{Acknowledgements}
This information has been authored by employees of the Los Alamos National 
Security, LLC (LANS) operator of Los Alamos National Laboratory under contract 
No DE-AC52-06NA25396 with the U.S. Department of Energy.

\pagebreak
\bibliographystyle{style/ans_js}                                                                           %custom ANS journal submission template bibliography style
\bibliography{bibliography}

\end{document}


