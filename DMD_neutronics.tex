\documentclass[12pt]{article}
\usepackage{afterpage}
\usepackage{amsmath}
\usepackage{amsfonts}
\usepackage{amssymb}
\usepackage{amsbsy}
\usepackage{bm}
\usepackage{epsfig}
\usepackage{rotating}
\usepackage{setspace}
\usepackage{tabls}
\usepackage{hhline}
\usepackage{float}
\usepackage{subfigure}
%\usepackage{subfigmat}
\usepackage{citesort}
%\usepackage{cites}
\usepackage{overcite}
                                                                                                            
% uncomment for submission of manuscript to NSE
\usepackage[nolists, nomarkers]{endfloat}
%
% use to include postscript figures
\usepackage{graphicx}
%
%\usepackage[light,firsttwo]{draftcopy}
%\draftcopySetGrey{0.90}
                                                                                                            
%\usepackage{dbl}

% -----------------------------------------------------------------------------
% define newcommands
% -----------------------------------------------------------------------------

%\setlength{\floatsep}{4pt plus 1pt minus 1pt}
\setlength{\textfloatsep}{8pt plus 1pt minus 1pt}
%\setlength{\intextsep}{4pt plus 1pt minus 1pt}
\setlength{\abovedisplayskip}{4pt plus 1pt minus 1pt}
\setlength{\belowdisplayskip}{4pt plus 1pt minus 1pt}

\makeatletter
\renewcommand{\@thesubfigure}{\thefigure\thesubfigure\space}
\makeatother
% =================================================================================================
% more new commands
% +++++++++++++++++++++++++++++++++++++++++++++++++++++++++++++++++++++++++++++++++++++++++++++++++
\setlength{\textwidth}{6.5in}
\setlength{\textheight}{8.5in}
\setlength{\oddsidemargin}{0in}
\setlength{\topmargin}{0pt}
\setlength{\headsep}{12pt}
%\addtolength{\oddsidemargin}{-0.5in}
%\addtolength{\textwidth}{1.0in}
%\addtolength{\textheight}{1.0in}
\renewcommand{\thefootnote}{\fnsymbol{footnote}}
%
% -----------------------------------------------------------------------------
% define newcommands
% -----------------------------------------------------------------------------

%\setlength{\floatsep}{4pt plus 1pt minus 1pt}
\setlength{\textfloatsep}{8pt plus 1pt minus 1pt}
%\setlength{\intextsep}{4pt plus 1pt minus 1pt}
\setlength{\abovedisplayskip}{4pt plus 1pt minus 1pt}
\setlength{\belowdisplayskip}{4pt plus 1pt minus 1pt}

% =================================================================================================
% more new commands
% +++++++++++++++++++++++++++++++++++++++++++++++++++++++++++++++++++++++++++++++++++++++++++++++++

% Ways of grouping things
%
\newcommand{\bracket}[1]{\left[ #1 \right]}
\newcommand{\bracet}[1]{\left\{ #1 \right\}}
\newcommand{\fn}[1]{\left( #1 \right)}
\newcommand{\ave}[1]{\left\langle #1 \right\rangle}
%
% Derivative forms
%
\newcommand{\dx}[1]{\,d#1}
\newcommand{\dxdy}[2]{\frac{\partial #1}{\partial #2}}
\newcommand{\dxdt}[1]{\frac{\partial #1}{\partial t}}
\newcommand{\dxdz}[1]{\frac{\partial #1}{\partial z}}
\newcommand{\dfdt}[1]{\frac{\partial}{\partial t} \fn{#1}}
\newcommand{\dfdz}[1]{\frac{\partial}{\partial z} \fn{#1}}
\newcommand{\ddt}[1]{\frac{\partial}{\partial t} #1}
\newcommand{\ddz}[1]{\frac{\partial}{\partial z} #1}
\newcommand{\dd}[2]{\frac{\partial}{\partial #1} #2}
\newcommand{\ddx}[1]{\frac{\partial}{\partial x} #1}
\newcommand{\ddy}[1]{\frac{\partial}{\partial y} #1}
%
% Vector forms
%
%\renewcommand{\vec}[1]{\ensuremath{\stackrel{\rightarrow}{#1}}}
%\renewcommand{\div}{\ensuremath{\vec{\nabla} \cdot}}
%\newcommand{\grad}{\ensuremath{\vec{\nabla}}}
\renewcommand{\vec}[1]{\overrightarrow{#1}}
\renewcommand{\div}{\vec{\nabla}\! \cdot \!}
\newcommand{\grad}{\vec{\nabla}}
\newcommand{\oa}[1]{\fn{\frac{1}{3}\hat{\Omega}\!\cdot\!\overrightarrow{A_{#1}}}}

%
% Equation beginnings and endings
%
\newcommand{\bea}{\begin{eqnarray}}
\newcommand{\eea}{\end{eqnarray}}
\newcommand{\be}{\begin{equation}}
\newcommand{\ee}{\end{equation}}
\newcommand{\beas}{\begin{eqnarray*}}
\newcommand{\eeas}{\end{eqnarray*}}
\newcommand{\bdm}{\begin{displaymath}}
\newcommand{\edm}{\end{displaymath}}
%
% Equation punctuation
%
\newcommand{\pec}{\hspace{0.25in},}
\newcommand{\pep}{\hspace{0.25in}.}
\newcommand{\pev}{\hspace{0.25in}}
%
% Equation labels and references, figure references, table references
%
\newcommand{\LEQ}[1]{\label{eq:#1}}
\newcommand{\EQ}[1]{Eq.~(\ref{eq:#1})}
\newcommand{\EQS}[1]{Eqs.~(\ref{eq:#1})}
\newcommand{\REQ}[1]{\ref{eq:#1}}
\newcommand{\LFI}[1]{\label{fi:#1}}
\newcommand{\FI}[1]{Fig.~\ref{fi:#1}}
\newcommand{\RFI}[1]{\ref{fi:#1}}
\newcommand{\LTA}[1]{\label{ta:#1}}
\newcommand{\TA}[1]{Table~\ref{ta:#1}}
\newcommand{\RTA}[1]{\ref{ta:#1}}

%
% List beginnings and endings
%
\newcommand{\bl}{\bss\begin{itemize}}
\newcommand{\el}{\vspace{-.5\baselineskip}\end{itemize}\ess}
\newcommand{\ben}{\bss\begin{enumerate}}
\newcommand{\een}{\vspace{-.5\baselineskip}\end{enumerate}\ess}
%
% Figure and table beginnings and endings
%
\newcommand{\bfg}{\begin{figure}}
\newcommand{\efg}{\end{figure}}
\newcommand{\bt}{\begin{table}}
\newcommand{\et}{\end{table}}
%
% Tabular and center beginnings and endings
%
\newcommand{\bc}{\begin{center}}
\newcommand{\ec}{\end{center}}
\newcommand{\btb}{\begin{center}\begin{tabular}}
\newcommand{\etb}{\end{tabular}\end{center}}
%
% Single space command
%
%\newcommand{\bss}{\begin{singlespace}}
%\newcommand{\ess}{\end{singlespace}}
\newcommand{\bss}{\singlespacing}
\newcommand{\ess}{\doublespacing}
%
%---New environment "arbspace". (modeled after singlespace environment
%                                in Doublespace.sty)
%   The baselinestretch only takes effect at a size change, so do one.
%
\def\arbspace#1{\def\baselinestretch{#1}\@normalsize}
\def\endarbspace{}
\newcommand{\bas}{\begin{arbspace}}
\newcommand{\eas}{\end{arbspace}}
%
% An explanation for a function
%
\newcommand{\explain}[1]{\mbox{\hspace{2em} #1}}
%
% Quick commands for symbols
%
\newcommand{\half}{\frac{1}{2}}
\newcommand{\third}{\frac{1}{3}}
\newcommand{\twothird}{\frac{2}{3}}
\newcommand{\mdot}{\dot{m}}
\newcommand{\ten}[1]{\times 10^{#1}\,}
\newcommand{\cL}{{\cal L}}
\newcommand{\cD}{{\cal D}}
\newcommand{\cF}{{\cal F}}
\newcommand{\cE}{{\cal E}}
\renewcommand{\Re}{\mbox{Re}}
\newcommand{\Ma}{\mbox{Ma}}
%
% Inclusion of Graphics Data
%
%\input{psfig}
%\psfiginit
%
% More Quick Commands
%
\newcommand{\bi}{\begin{itemize}}
\newcommand{\ei}{\end{itemize}}
\newcommand{\dxi}{\Delta x_i}
\newcommand{\dyj}{\Delta y_j}
\newcommand{\ts}[1]{\textstyle #1}
% =================================================================================================
\date{}

\begin{document}

%\bibliographystyle{nse}
%\bibnum{p}

\thispagestyle{empty}

\bss
\bc
{\Large \bf Dynamic Mode Decomposition for Subcritical Metal Systems}\\
\vspace{0.3in}
{\large Zachary Hardy and Jim E. Morel\\
Texas A\&M University\\
Department of Nuclear Engineering\\
TAMU 3133\\
College Station, TX 77843-3133\\
$ $\\
Cory Ahrens\\
Los Alamos National Laboratory\\
Los Alamos, NM  87545\\
$ $\\
}
\emph{zach.hardy@tamu.edu, morel@tamu.edu, cdahrens@lanl.gov}

\vspace{1.5in}

Send proofs and page charges to:\\
\vspace{0.1in}
Dr. Jim Morel\\
Texas A\&M University\\
Department of Nuclear Engineering\\
TAMU 3133\\
College Station, TX 77843-3133\\
\vspace{0.25in}
25 Pages -- 1 Table -- 4 Figures \\
\ec
\ess

% =================================================================================================

\newpage

\begin{abstract}
In this paper we explore the use of Dynamic Mode Decomposition (DMD) for modeling the kinetics of 
subcritical metal systems pulsed with fast neutrons.  Our ultimate purpose is to obtain a fast and 
accurate reduced-order model for such systems that can be used to develop an emulator.  
An alternative to DMD is $\alpha$-eigenfunction expansions, but we show that DMD is vastly superior in 
several ways for the systems of interest to us. 
\end{abstract}

\section{Introduction}
The modeling of subcritical metal systems subjected to short bursts of fast neutrons is of interest 
to us.  While such modeling can certainly be done with a time-dependent Sn code, we want to develop a
reduced-order model that can provide insight into such systems and form the basis of an emulator in 
order to solve inverse problems quickly and compute sensitivities economically.  For a supercritical 
system, it is clear that the transport solution will eventually be completely dominated by the 
fundamental mode $\alpha$-eigenfunction because it grows more quickly than any other mode.  However, the 
situation is quite different for a subcritical system because the fundamental mode $\alpha$-eigenfunction 
is simply the most slowly decaying mode.  There is no guarantee that it will dominate or even make a significant 
contribution to the solution before the solution is negligibly small.  Its practical importance in a solution 
depends upon the extent to which it is present in the expansion for the initial condition.  In 
a subcritical system the fundamental mode will be dominated by the slowest neutrons, which will be thermal 
neutrons. In a metal system subjected to a burst of fast neutrons, there will be almost no thermal neutrons 
created before the pulse has effectively died away.  Thus it seems likely that an $\alpha$-eigenfunction expansion 
will be particularly inefficient for such systems, because a high degree of cancellation between the eigenfunctions 
will be required to achieve a negligible thermal neutron component in the initial condition.  

Dynamic Mode Decomposition (DMD) is a reduced order technique for modeling dynamical systems.  It has been applied 
in many areas (need references) and is perhaps best known for its use in the fluid-flow community.  The purpose of 
this paper is to perform a preliminary investigation of DMD as an alternative to $\alpha$-eigenfunction expansions 
for modeling our pulsed neutron experiments. (Need to reference any previous neutronics work with DMD here.) 
For this initial study, we use a very approximate but relevant 
high-dimensional model consisting of a time-dependent 1-D spherical geometry three-group diffusion approximation. 
We have a analytic eigenfunction solution for these equations, as well as a computer code for solving 
these equations with second-order accuracy in both time and space.  We later describe DMD in detail, but at 
this point it suffices to say that given a time series of vector ``snapshots'' from a simulation or experiment, 
DMD produces a time-dependent solution for those snapshots that is constructed from a sum of snapshot modes, each with 
an exponential decay rate.  For instance, the snapshots could be space-dependent three-group diffusion scalar fluxes 
from a time-dependent calculation.  They could also be an analogous time series of any quantity of interest computed 
from the diffusion solution, such as space-dependent reaction rates.  The form of the DMD solution is identical to that 
of an alpha-eigenfunction solution.  However, the DMD modes can only contain what the snapshots contain, so if the snapshots 
lack a thermal neutron component, so will the DMD modes.  This suggests that fewer DMD modes should be required for 
our problems than $\alpha$-eigenfunctions.

Indeed we present results demonstrating that the DMD method is extremely accurate and far more efficient than 
$\alpha$-eigenfunction expansions for our problems.  The remainder of this paper is organized as follows.  First we describe 
the particular variant of the DMD method that we use.  Then we describe the three-group diffusion model, followed by a 
description of our space-time discretization for the diffusion model.  Finally, numerical results are given, followed by 
conclusions and recommendations for future work.


\section{The DMD Method}

\section{The Three-Group Diffusion Model}

\section{Discretization of the Diffusion Equations}

\section{Numerical Results}

\section{Conclusions and Recommendations for Future Work}

%\begin{thebibliography}{99}

%\end{thebibliography}

\end{document} 

